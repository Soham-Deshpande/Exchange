\documentclass[12pt]{article}
\usepackage{amsthm}
\usepackage{amssymb}
\theoremstyle{definition}
\newtheorem{definition}{Definition}[section]


%\usepackage{lingmacros}
%\usepackage{tree-dvips}
\begin{document}

\title{Designing an Exchange}
\author{Soham Deshpande and Ellis Walker}
\maketitle
\clearpage

\tableofcontents

\clearpage
\section{Introduction}
The aim of this project is to provide insight into the inner workings of an 
exchange as well as the orderbook matching algorithms used. I have chosen
to conduct this in C to force myself to learn the language. C does also provide
benefits such as fast execution speeds and direct access to memory, both which
are core to creating a good exchange. This documentation will discuss the 
choices made as well explanations of the relevant algorithms used. Eventually,
a server will be set up to allow multiple agents to trade, mimicking a real
exchange. 
  

\section{Order Types}
We have decided to select a few order types that can be excecuted, these reflect
the most common trades made and will allow the user to have enough flexibilty
to be useful. Though not in surplus of features, the aim of this project is on
the order matching algorithm, not on the wants of the user.
\\
The different types being offered include:
\begin{enumerate}
  \item Market Orders
  \item Limit Orders
  \item Stop-loss Orders
  \item Stop-limit Orders
  \item Immediate or Cancel
  \item Fill or Kill
  \item Good 'Til Cancelled
\end{enumerate}

\subsubsection{Market Orders}
These are the most trivial of orders, simply buying or selling at or near the 
posted price. These guarantee a position in market and will be excecuted as 
soon as possible. 

\subsubsection{Limit Orders}
Limit orders can be thought of as pending orders that are only executed once a 
certain condition or limit is reached. This limit is based on the price of the
security being traded. The order will be cancelled if the price is not met 
whilst trying to sell or buy at the pre-determined level. 4 types of Limit 
orders exist: Buy limit, Sell limit, Buy stop, and Sell stop. 
\\
Buy Limits allow you to purchase a given security at or below the specified 
price. To complement that, the Sell limit allows you to sell the given security
at or above the specified price. 
Buy stops allow you to buy a given security at a price above the current
market bid price. Therefore the Sell stop is in place to allow a security to 
be sold at a price below the current market ask.

\subsubsection{Stop-loss Orders}
This order type, unlike Market and Limit orders, remain inactive until a certain
price threshold is met. At this point is behaves like a market order.

\subsubsection{Immediate or Cancel}
This order dictates that the whatever amount of an order can be fulfilled must 
be done as soon as possible with the rest of the order being cancelled.


\subsubsection{Fill or Kill}
This one is similar to Immediate or Cancel however either the entire order must
be executed in a short amount of time otherwise be cancelled. 


\subsubsection{Good 'Till Cancelled}
This order applies a time restriction so that the order will remain active 
until the order is cancelled by the user.


\section{Orderbook Design}
The orderbook design is crucial, one we must consider carefully encorporating
the use case for our exchange. The two main branches we could explore is 
Central Limit Order Book (CLOB) or the Automated Market Maker Model (AMM). 

\subsection{Central Limit Order Book} 
A limit order book is a method of containing all orders sent to the market,
organised by the sign of the order, time stamp, quanitity and other relevant 
factors. A LOB will contains all the information available on a  specific market
and reflects the movements of the market under influence of all 
heterogeneous agents involved. These faciliate trades between agents, matching 
relevant buyers to sellers. In the age of electronic trading, pressure falls on 
exchanges to provide real-time access to the LOB to market participants. The
following sections will discuss and mathematically describle LOB design
and dynamics.


\subsection{Automated Market Maker Model}


\section{Central Limit Order Book}
A central limit order book can be described precisely using mathematics, 
providing a platform to conduct analysis upon.\\
Firstmost, let's define an order sent to the exchange as: 
\begin{definition}
An order $x = (p_x,\omega_x,t_x)$ can be defined as an order submitted at time 
$t_x$ with price $p_x$  and size $\omega_x > 0$. This can be seen as a commitment to 
 sell up to $| \omega_x |$ units at a price no lower than $p_x$. Respectively,
a buy order can be interpreted from a size of $\omega_x < 0$.  
\end{definition}
Secondly we need to define the lot size and tick size in the LOB. 
 
 \begin{definition}
 The lot size $\sigma$ of an LOB is the smallest amount of a given asset that
   can be traded within it. Orders must be sent as multiples of $\sigma$, given 
   as size $\omega_x \in \{\pm k\sigma | k = 1,2,3 ...\}$.
 \end{definition}

 \begin{definition}
 The tick size $\pi$ of an LOB is the smallest denomination of price movements 
   allowed for a particular security. All orders submitted must adhere to an 
   accuracy of $\pi$.
 \end{definition}

Lastly, we can define the fundemental platform, the LOB.

\begin{definition}
  An LOB $\mathcal{L}$ is the set of all active orders in a market at time $t$.
\end{definition}

We can proceed to explore the terms bid price, ask price, mid price and bid-ask 
spread, all crucial to understanding the dynamics of an LOB. 

\subsection{Equilibrium}

\subsection{Spread}

\section{Orderbook Matching}

\section{Exchange Design}

\section{Networking}

\section{C Code}

\end{document}
