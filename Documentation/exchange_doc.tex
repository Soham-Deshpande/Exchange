\documentclass[12pt]{article}
%\usepackage{lingmacros}
%\usepackage{tree-dvips}
\begin{document}

\title{Designing an Exchange}
\author{Soham Deshpande and Ellis Walker}
\maketitle
\clearpage

\tableofcontents

\clearpage
\section{Introduction}
The aim of this project is to provide insight into the inner workings of an 
exchange as well as the orderbook matching algorithms used. I have chosen
to conduct this in C to force myself to learn the language. C does also provide
benefits such as fast execution speeds and direct access to memory, both which
are core to creating a good exchange. This documentation will discuss the 
choices made as well explanations of the relevant algorithms used. Eventually,
a server will be set up to allow multiple agents to trade, mimicking a real
exchange. 
  

\section{Order Types}
We have decided to select a few order types that can be excecuted, these reflect
the most common trades made and will allow the user to have enough flexibilty
to be useful. Though not in surplus of features, the aim of this project is on
the order matching algorithm, not on the wants of the user.
\\
The different types being offered include:
\begin{enumerate}
  \item Market Orders
  \item Limit Orders
  \item Stop-loss Orders
  \item Stop-limit Orders
  \item Immediate or Cancel
  \item Fill or Kill
  \item Good 'Til Cancelled
\end{enumerate}

\subsubsection{Market Orders}
These are the most trivial of orders, simply buying or selling at or near the 
posted price. These guarantee a position in market and will be excecuted as 
soon as possible. 

\subsubsection{Limit Orders}
Limit orders can be thought of as pending orders that are only executed once a 
certain condition or limit is reached. This limit is based on the price of the
security being traded. The order will be cancelled if the price is not met 
whilst trying to sell or buy at the pre-determined level. 4 types of Limit 
orders exist: Buy limit, Sell limit, Buy stop, and Sell stop. 
\\
Buy Limits allow you to purchase a given security at or below the specified 
price. To complement that, the Sell limit allows you to sell the given security
at or above the specified price. 
Buy stops allow you to buy a given security at a price above the current
market bid price. Therefore the Sell stop is in place to allow a security to 
be sold at a price below the current market ask.

\subsubsection{Stop-loss Orders}
This order type, unlike Market and Limit orders, remain inactive until a certain
price threshold is met. At this point is behaves like a market order.

\subsubsection{Immediate or Cancel}
This order dictates that the whatever amount of an order can be fulfilled must 
be done as soon as possible with the rest of the order being cancelled.


\subsubsection{Fill or Kill}
This one is similar to Immediate or Cancel however either the entire order must
be executed in a short amount of time otherwise be cancelled. 


\subsubsection{Good 'Till Cancelled}
This order applies a time restriction so that the order will remain active 
until the order is cancelled by the user.


\section{Orderbook Design}
The orderbook design is crucial, one we must consider carefully encorporating
the use case for our exchange. The two main branches we could explore is 
Central Limit Order Book (CLOB) or the Automated Market Maker Model (AMM). 

\subsection{Central Limit Order Book} 

\subsection{Automated Market Maker Model}


\section{Central Limit Order Book}

\subsection{Equilibrium}

\subsection{Spread}

\section{Orderbook Matching}

\section{Exchange Design}

\section{Networking}

\section{C Code}

\end{document}
